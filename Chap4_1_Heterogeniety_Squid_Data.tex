% Options for packages loaded elsewhere
\PassOptionsToPackage{unicode}{hyperref}
\PassOptionsToPackage{hyphens}{url}
%
\documentclass[
]{article}
\usepackage{lmodern}
\usepackage{amssymb,amsmath}
\usepackage{ifxetex,ifluatex}
\ifnum 0\ifxetex 1\fi\ifluatex 1\fi=0 % if pdftex
  \usepackage[T1]{fontenc}
  \usepackage[utf8]{inputenc}
  \usepackage{textcomp} % provide euro and other symbols
\else % if luatex or xetex
  \usepackage{unicode-math}
  \defaultfontfeatures{Scale=MatchLowercase}
  \defaultfontfeatures[\rmfamily]{Ligatures=TeX,Scale=1}
\fi
% Use upquote if available, for straight quotes in verbatim environments
\IfFileExists{upquote.sty}{\usepackage{upquote}}{}
\IfFileExists{microtype.sty}{% use microtype if available
  \usepackage[]{microtype}
  \UseMicrotypeSet[protrusion]{basicmath} % disable protrusion for tt fonts
}{}
\makeatletter
\@ifundefined{KOMAClassName}{% if non-KOMA class
  \IfFileExists{parskip.sty}{%
    \usepackage{parskip}
  }{% else
    \setlength{\parindent}{0pt}
    \setlength{\parskip}{6pt plus 2pt minus 1pt}}
}{% if KOMA class
  \KOMAoptions{parskip=half}}
\makeatother
\usepackage{xcolor}
\IfFileExists{xurl.sty}{\usepackage{xurl}}{} % add URL line breaks if available
\IfFileExists{bookmark.sty}{\usepackage{bookmark}}{\usepackage{hyperref}}
\hypersetup{
  pdftitle={Dealing with Heterogeniety},
  hidelinks,
  pdfcreator={LaTeX via pandoc}}
\urlstyle{same} % disable monospaced font for URLs
\usepackage[margin=1in]{geometry}
\usepackage{color}
\usepackage{fancyvrb}
\newcommand{\VerbBar}{|}
\newcommand{\VERB}{\Verb[commandchars=\\\{\}]}
\DefineVerbatimEnvironment{Highlighting}{Verbatim}{commandchars=\\\{\}}
% Add ',fontsize=\small' for more characters per line
\usepackage{framed}
\definecolor{shadecolor}{RGB}{248,248,248}
\newenvironment{Shaded}{\begin{snugshade}}{\end{snugshade}}
\newcommand{\AlertTok}[1]{\textcolor[rgb]{0.94,0.16,0.16}{#1}}
\newcommand{\AnnotationTok}[1]{\textcolor[rgb]{0.56,0.35,0.01}{\textbf{\textit{#1}}}}
\newcommand{\AttributeTok}[1]{\textcolor[rgb]{0.77,0.63,0.00}{#1}}
\newcommand{\BaseNTok}[1]{\textcolor[rgb]{0.00,0.00,0.81}{#1}}
\newcommand{\BuiltInTok}[1]{#1}
\newcommand{\CharTok}[1]{\textcolor[rgb]{0.31,0.60,0.02}{#1}}
\newcommand{\CommentTok}[1]{\textcolor[rgb]{0.56,0.35,0.01}{\textit{#1}}}
\newcommand{\CommentVarTok}[1]{\textcolor[rgb]{0.56,0.35,0.01}{\textbf{\textit{#1}}}}
\newcommand{\ConstantTok}[1]{\textcolor[rgb]{0.00,0.00,0.00}{#1}}
\newcommand{\ControlFlowTok}[1]{\textcolor[rgb]{0.13,0.29,0.53}{\textbf{#1}}}
\newcommand{\DataTypeTok}[1]{\textcolor[rgb]{0.13,0.29,0.53}{#1}}
\newcommand{\DecValTok}[1]{\textcolor[rgb]{0.00,0.00,0.81}{#1}}
\newcommand{\DocumentationTok}[1]{\textcolor[rgb]{0.56,0.35,0.01}{\textbf{\textit{#1}}}}
\newcommand{\ErrorTok}[1]{\textcolor[rgb]{0.64,0.00,0.00}{\textbf{#1}}}
\newcommand{\ExtensionTok}[1]{#1}
\newcommand{\FloatTok}[1]{\textcolor[rgb]{0.00,0.00,0.81}{#1}}
\newcommand{\FunctionTok}[1]{\textcolor[rgb]{0.00,0.00,0.00}{#1}}
\newcommand{\ImportTok}[1]{#1}
\newcommand{\InformationTok}[1]{\textcolor[rgb]{0.56,0.35,0.01}{\textbf{\textit{#1}}}}
\newcommand{\KeywordTok}[1]{\textcolor[rgb]{0.13,0.29,0.53}{\textbf{#1}}}
\newcommand{\NormalTok}[1]{#1}
\newcommand{\OperatorTok}[1]{\textcolor[rgb]{0.81,0.36,0.00}{\textbf{#1}}}
\newcommand{\OtherTok}[1]{\textcolor[rgb]{0.56,0.35,0.01}{#1}}
\newcommand{\PreprocessorTok}[1]{\textcolor[rgb]{0.56,0.35,0.01}{\textit{#1}}}
\newcommand{\RegionMarkerTok}[1]{#1}
\newcommand{\SpecialCharTok}[1]{\textcolor[rgb]{0.00,0.00,0.00}{#1}}
\newcommand{\SpecialStringTok}[1]{\textcolor[rgb]{0.31,0.60,0.02}{#1}}
\newcommand{\StringTok}[1]{\textcolor[rgb]{0.31,0.60,0.02}{#1}}
\newcommand{\VariableTok}[1]{\textcolor[rgb]{0.00,0.00,0.00}{#1}}
\newcommand{\VerbatimStringTok}[1]{\textcolor[rgb]{0.31,0.60,0.02}{#1}}
\newcommand{\WarningTok}[1]{\textcolor[rgb]{0.56,0.35,0.01}{\textbf{\textit{#1}}}}
\usepackage{graphicx}
\makeatletter
\def\maxwidth{\ifdim\Gin@nat@width>\linewidth\linewidth\else\Gin@nat@width\fi}
\def\maxheight{\ifdim\Gin@nat@height>\textheight\textheight\else\Gin@nat@height\fi}
\makeatother
% Scale images if necessary, so that they will not overflow the page
% margins by default, and it is still possible to overwrite the defaults
% using explicit options in \includegraphics[width, height, ...]{}
\setkeys{Gin}{width=\maxwidth,height=\maxheight,keepaspectratio}
% Set default figure placement to htbp
\makeatletter
\def\fps@figure{htbp}
\makeatother
\setlength{\emergencystretch}{3em} % prevent overfull lines
\providecommand{\tightlist}{%
  \setlength{\itemsep}{0pt}\setlength{\parskip}{0pt}}
\setcounter{secnumdepth}{-\maxdimen} % remove section numbering
\ifluatex
  \usepackage{selnolig}  % disable illegal ligatures
\fi

\title{Dealing with Heterogeniety}
\author{}
\date{\vspace{-2.5em}}

\begin{document}
\maketitle

This page is a reproducible work for Chapter 4.1 of Zuur et al.~(2009).
This particular example is useful to practice methods to address
heterogeniety (heteroscedasticy) in linear regression models.

One of the important assumptions in linear regression is
``homeoscedasticy (homeogeneity) of variance'', which means that the
spread of dataset is same at each X values. (i.e.~the spread of reponse
is the same for along the range of predictor variable).

One way to deal with heterogeniety is `a mean-variance stabilisng'
transformation. Formal test for homeogeniety require normality of the
data (e.g.~Barlett's test). Zuur et al (2009) suggest assessing
homoegeneity based on `graphical inspection' of residuals (pp20).

Serious heterogeniety could cause major harm to invalidate the outcome
of iinear regression analysis. Ignoring such issue could leave the
regression parameters with incorrect standard errors (Zuur et al.~2009),
which leads to incorrect distribution of statistics (e.g., F / t
statistic is no longer F / t distributed), thus harming statistical
significance of the tests.

With extra Mathematical effort, heterogeniety can be incorporated to
models and can provide extra biological information (Zuur et al.~2009)

\begin{Shaded}
\begin{Highlighting}[]
\NormalTok{M1}\OtherTok{\textless{}{-}} \FunctionTok{lm}\NormalTok{ (Testisweight }\SpecialCharTok{\textasciitilde{}}\NormalTok{DML}\SpecialCharTok{*}\NormalTok{fMONTH, }\AttributeTok{data=}\NormalTok{Squid)}
\NormalTok{E0}\OtherTok{\textless{}{-}}\FunctionTok{resid}\NormalTok{(M1)}
\FunctionTok{coplot}\NormalTok{(E0}\SpecialCharTok{\textasciitilde{}}\NormalTok{DML}\SpecialCharTok{|}\NormalTok{fMONTH,}\AttributeTok{ylab=}\StringTok{"lm residuals"}\NormalTok{, }\AttributeTok{data=}\NormalTok{Squid)}
\end{Highlighting}
\end{Shaded}

\includegraphics{Chap4_1_Heterogeniety_Squid_Data_files/figure-latex/unnamed-chunk-1-1.pdf}

\begin{Shaded}
\begin{Highlighting}[]
\NormalTok{M1}\OtherTok{\textless{}{-}}\FunctionTok{gls}\NormalTok{(Testisweight }\SpecialCharTok{\textasciitilde{}}\NormalTok{ DML}\SpecialCharTok{*}\NormalTok{fMONTH, }\AttributeTok{data=}\NormalTok{Squid)}
\NormalTok{E0\_n}\OtherTok{\textless{}{-}}\FunctionTok{resid}\NormalTok{(M1, }\AttributeTok{type=}\StringTok{"normalized"}\NormalTok{)}
\FunctionTok{coplot}\NormalTok{(E0\_n}\SpecialCharTok{\textasciitilde{}}\NormalTok{DML}\SpecialCharTok{|}\NormalTok{fMONTH,}\AttributeTok{ylab=}\StringTok{"normalized lm residuals"}\NormalTok{, }\AttributeTok{data=}\NormalTok{Squid)}
\end{Highlighting}
\end{Shaded}

\includegraphics{Chap4_1_Heterogeniety_Squid_Data_files/figure-latex/unnamed-chunk-2-1.pdf}

\begin{Shaded}
\begin{Highlighting}[]
\DocumentationTok{\#\#\# Ordinary residuals}

\NormalTok{vf4}\OtherTok{\textless{}{-}} \FunctionTok{varPower}\NormalTok{(}\AttributeTok{form =}\SpecialCharTok{\textasciitilde{}}\NormalTok{DML}\SpecialCharTok{|}\NormalTok{ fMONTH)}
\NormalTok{M.gls4}\OtherTok{\textless{}{-}}\FunctionTok{gls}\NormalTok{(Testisweight }\SpecialCharTok{\textasciitilde{}}\NormalTok{ DML}\SpecialCharTok{*}\NormalTok{fMONTH, }\AttributeTok{data=}\NormalTok{Squid, }\AttributeTok{weights=}\NormalTok{vf4)}
\NormalTok{E1}\OtherTok{\textless{}{-}}\FunctionTok{resid}\NormalTok{(M.gls4)}
\FunctionTok{coplot}\NormalTok{(E1}\SpecialCharTok{\textasciitilde{}}\NormalTok{ DML }\SpecialCharTok{|}\NormalTok{ fMONTH, }\AttributeTok{ylab =}\StringTok{"Ordinary residuals for power of the covariate"}\NormalTok{, }\AttributeTok{data=}\NormalTok{ Squid)}
\end{Highlighting}
\end{Shaded}

\includegraphics{Chap4_1_Heterogeniety_Squid_Data_files/figure-latex/unnamed-chunk-3-1.pdf}

\begin{Shaded}
\begin{Highlighting}[]
\DocumentationTok{\#\#\# Ordinary residuals}
\NormalTok{E2}\OtherTok{\textless{}{-}}\FunctionTok{resid}\NormalTok{(M.gls4,}\AttributeTok{type=}\StringTok{"normalized"}\NormalTok{)}
\FunctionTok{coplot}\NormalTok{(E2}\SpecialCharTok{\textasciitilde{}}\NormalTok{ DML }\SpecialCharTok{|}\NormalTok{ fMONTH, }\AttributeTok{ylab =}\StringTok{"Normalized residuals"}\NormalTok{, }\AttributeTok{data=}\NormalTok{ Squid)}
\end{Highlighting}
\end{Shaded}

\includegraphics{Chap4_1_Heterogeniety_Squid_Data_files/figure-latex/unnamed-chunk-4-1.pdf}

\end{document}
